%--------------------------------------------------%
% generated by the codebookr R package
% created by Joshua C. Fjelstul, Ph.D.
%--------------------------------------------------%

\documentclass[10pt]{article}

%--------------------------------------------------%
% packages
%--------------------------------------------------%

% page layout
\usepackage{geometry}

% fonts
\usepackage[english]{babel}
\usepackage{underscore}
\usepackage{anyfontsize}
\usepackage[utf8]{inputenc}
\usepackage[T1]{fontenc}
\usepackage{fontspec}

% graphics and tables
\usepackage{graphicx} % add figures
\usepackage{xcolor} % change font color
\usepackage{tikz} % add graphics

% paragraph spacing
\usepackage{setspace}

% hyperlinks
\usepackage{url}

% table of contents
\usepackage{tocloft}

% test alignment
\usepackage{ragged2e}

% multi-page tables
\usepackage{longtable}

% custom lists
\usepackage{enumitem}

% insert content on every page
\usepackage{atbegshi} 

% code formatting
\usepackage{tcolorbox}

%--------------------------------------------------%
% colors
%--------------------------------------------------%

% define colors
\definecolor{themecolor}{HTML}{4D9FEB}
\definecolor{background}{HTML}{EEF6FD}

% format hyperlinks
\usepackage[colorlinks=true,linkcolor=themecolor,citecolor=themecolor,urlcolor=themecolor,breaklinks=true]{hyperref}

%--------------------------------------------------%
% formatting
%--------------------------------------------------%

% configure main font
\setmainfont[Ligatures=TeX,BoldFont={Roboto Medium}]{Roboto Light}
\setmonofont[Ligatures=TeX]{Roboto Mono-Light}

% set page margins
\geometry{top = 1.5in, bottom = 1.5in, left = 1.5in, right = 1.5in}

% set paper size
\geometry{letterpaper}

% format table of contents
\renewcommand{\cftsecdotsep}{10}
\renewcommand{\cftsecleader}{\cftdotfill{\cftdotsep}}
\renewcommand{\cftsecfont}{{\small\color{black!75}\bfseries}}
\renewcommand{\cftsecpagefont}{{\small\color{black!75}\normalfont}}

% adjust spacing
\usepackage{parskip}
\parskip=10pt
\renewcommand{\baselinestretch}{1.4}

% hyphen formatting
\hyphenpenalty = 10000
\exhyphenpenalty = 10000

% prevent widow and orphan lines
\widowpenalty10000
\clubpenalty10000

%--------------------------------------------------%
% page elements
%--------------------------------------------------%

% a command to make a code box
\newtcbox{\codebox}{nobeforeafter,tcbox raise base,colback=black!5,colframe=white,coltext=black!75,boxrule=0pt,arc=3pt,boxsep=0pt,
left=4pt,right=4pt,top=3pt,bottom=3pt}

% a command to make a chip
\newtcbox{\chip}{nobeforeafter,tcbox raise base,colback=black!5,colframe=white,coltext=black!75,boxrule=0pt,arc=11pt,boxsep=0pt,
left=10pt,right=10pt,top=8pt,bottom=8pt}

% command to format code
\newcommand{\code}[1]{\codebox{{\footnotesize\texttt{#1}}}}

% command to highlight text
\newcommand{\highlight}[1]{{\color{themecolor} \textbf{#1}}}

% command to create a divider
\newcommand{\dividerline}{{\color{gray!10} \rule[4pt] {\textwidth}{3pt}}}

% command to add a cover
\newcommand{\cover}[4]{
\begin{tikzpicture}[remember picture,overlay, shift={(current page.south west)}]
\fill[themecolor] (0, 5.5in) rectangle ++ (8.5in, 5.5in); % header bar
\fill[black!5] (0, 4in) rectangle ++ (8.5in, 1.5in); % middle bar
\fill[white] (0, 0in) rectangle ++ (8.5in, 4in); % footer bar
\node[anchor=west] at (1.5in, 6.25in) {\color{white} \fontsize{60}{60}\selectfont \begin{minipage}{5.5in} \textbf{Codebook} \fontsize{15}{15}\selectfont \hspace{5pt} v #2 \end{minipage}};
\node[anchor=west, align=left] at (1.5in, 4.75in) {\begin{minipage}{5.5in} \color{black!40} \fontsize{#4}{#4} \selectfont #1 \end{minipage}};
\node[anchor=west, align=left, minimum height=2in] at (1.5in, 2.55in) {\begin{minipage}[t][2in]{5.5in} \color{black!40} \fontsize{10}{10} \selectfont #3 \end{minipage}};
\end{tikzpicture}
}

% command to add a header page
\newcommand{\headerpage}[4]{
	\newpage
	\begin{tikzpicture}[remember picture,overlay, shift={(current page.south west)}]
		\fill[themecolor] (0, 9in) rectangle ++ (8.5in, 2in); % header line 1
		\fill[black!5] (0, 8in) rectangle ++ (8.5in, 1in); % header line 2
		\node[anchor = west] at (1.5in, 9.6in) {\color{white} \fontsize{#3}{#3}\selectfont \textbf{#1}}; % heading
		\node[anchor = west] at (1.5in, 8.5in) {\color{black!40} \fontsize{#4}{#4}\selectfont #2}; % heading
	\end{tikzpicture}
	\phantomsection
	\addcontentsline{toc}{section}{#1}
	\vspace{1.5in}
}

% command to layout page
\newcommand\pagelayout{
	\begin{tikzpicture}[remember picture,overlay, shift={(current page.south west)}]
		% \fill[themecolor] (0, 10.75in) rectangle ++ (8.5in, 0.25in); % header
		\fill[black!5] (0, 0) rectangle ++ (8.5in, 0.5in); % footer
		\draw (0.25in, 0.25in) node[anchor = west] {\fontsize{9}{9}\selectfont \color{black!40} The EUIP Database Codebook \hspace{5pt} | \hspace{5pt} Joshua C. Fjelstul, Ph.D.}; % footer content
		\draw (8.25in, 0.25in) node[anchor = east] {\fontsize{9}{9}\selectfont \color{black!40} \thepage}; % page number
	\end{tikzpicture}
}

% add page layout 
\AtBeginShipout{
	\AtBeginShipoutUpperLeft{\pagelayout}
}

% command to add a subheading
\newcommand{\subheading}[1]{
\vspace{24pt}
{\color{themecolor} \fontsize{14}{14}\selectfont \textbf{#1}}
\vspace{6pt}
\dividerline
\vspace{-20pt}
}

%--------------------------------------------------%
% start document
%--------------------------------------------------%

\begin{document}

\clearpage
\pagestyle{empty}

\color{black!75}

\small

\begin{flushleft}

%--------------------------------------------------%
% cover
%--------------------------------------------------%

\cover{The European Union Infringement Procedure \\ (EUIP) Database}{1.0}{Joshua C. Fjelstul, Ph.D.}{16}

\newpage

%--------------------------------------------------%
% table of contents
%--------------------------------------------------%

% reset page counter
\setcounter{page}{1}

% format the table of contents header
% \renewcommand\contentsname{{\color{themecolor} \fontsize{14}{14}\selectfont Datasets}}
\renewcommand\contentsname{\subheading{Datasets} \vspace{0pt}}

% add the table of contents
\tableofcontents

% remove page number from table of contents pages
\addtocontents{toc}{\protect\thispagestyle{empty}}

\newpage

%--------------------------------------------------%
% content
%--------------------------------------------------%


%--------------------------------------------------%
% dataset
%--------------------------------------------------%

\headerpage{cases}{Case-level data}{30}{12}

\subheading{Description}

This dataset include data on infringement cases. There is one observation per case  (2002-2020). The dataset includes information on the department responsible for the case, the member state that the case is against, and the progression of the case through the infringement procedure.

\subheading{Variables}

\begin{description}[labelwidth=130pt, leftmargin=\dimexpr\labelwidth+\labelsep\relax, font=\normalfont, itemsep=10pt]
\item[\code{key\_id}] \code{numeric}\hspace{5pt}An ID number that uniquely identifies each observation in the dataset. 
\item[\code{case\_id}] \code{string}\hspace{5pt}An ID number that uniquely identifies each case.
\item[\code{case\_number}] \code{numeric}\hspace{5pt}The number of the case.
\item[\code{case\_year}] \code{numeric}\hspace{5pt}The year of the case.
\item[\code{member\_state\_id}] \code{numeric}\hspace{5pt}An ID number that uniquely identifies each member state. This ID number is assigned when member states are sorted by accession date and then alphabetically. 
\item[\code{member\_state}] \code{string}\hspace{5pt}The name of the member state that the Commission opened the case against. 
\item[\code{member\_state\_code}] \code{string}\hspace{5pt}A two letter code assigned by the Commission that uniquely identifies each member state. 
\item[\code{department\_id}] \code{numeric}\hspace{5pt}An ID number that uniquely identifies each Commission department.
\item[\code{department}] \code{string}\hspace{5pt}The name of the Commission department that opened the infringement case.
\item[\code{department\_code}] \code{string}\hspace{5pt}A multi-letter code assigned by the Commission that uniquely identifies each department.
\item[\code{directive}] \code{dummy}\hspace{5pt}A dummy variable indicating whether the description of the infringement case provided by the Commission references a directive that the case relates to. Directive numbers have the format \code{\#\#\#\#/\#\#\#\#}, where the first set of numbers is the year the directive was enacted and the second set is the number of the directive.
\item[\code{directive\_number}] \code{string}\hspace{5pt}The number of the directive that the infringement case relates to, if applicable. Coded \code{NA} if the description of the case provided by the Commission does not reference a specific directive number.
\item[\code{celex}] \code{string}\hspace{5pt}The CELEX number of the directive that the infringement case relates to, if applicable. Coded \code{NA} if the description of the case provided by the Commission does not reference a specific directive number. CELEX numbers for directives have the format \code{3\#\#\#\#L\#\#\#\#}, where the first set of numbers is the year the directive was enacted and the second set is the number of the directive.
\item[\code{case\_type\_id}] \code{numeric}\hspace{5pt}An ID number that uniquely identifies each type of state aid cases. Coded \code{1} for noncommunication cases, which are cases that relate to a member state failing to notify the Commission that it has transposed a directive by the stated deadline. Coded \code{2} for nonconformity cases, which are cases that relate to a member state incorrectly transposing a directive. 
\item[\code{case\_type}] \code{string}\hspace{5pt}The type of the infringement case. There are two types of cases. Coded \code{noncommunication} for cases that relate to a member state failing to notify the Commission that it has transposed a directive by the stated deadline. Coded \code{nonconformity} for cases that relate to a member state incorrectly transposing a directive. 
\item[\code{noncommunication}] \code{dummy}\hspace{5pt}A dummy variable indicating whether the infringement case is a noncommunication case, which is a case that relates to a member state failing to notify that Commission that it has transposed a directive by the stated deadline.
\item[\code{nonconformity}] \code{dummy}\hspace{5pt}A dummy variable indicating whether the infringement case is a nonconformity case, which is a case that relates to a member state incorrectly transposing a directive.
\item[\code{complete}] \code{dummy}\hspace{5pt}A dummy variable indicating whether the public records published by the Commission about the case are complete in that they include a record of a letter of formal notice under Article 258 of the Treaty of the Functioning of the European Union (TFEU), a record of the closing of the case, and a record of any stages of the infringement procedure in between. 
\item[\code{count\_press\_releases}] \code{numeric}\hspace{5pt}The number of press releases published by the Commission related to the case.
\item[\code{stage\_lfn\_258}] \code{dummy}\hspace{5pt}A dummy variable indicating whether the decision is a letter of formal notice sent by the Commission under Article 258 of the Treaty on the Functioning of the European Union (TFEU).
\item[\code{stage\_ro\_258}] \code{dummy}\hspace{5pt}A dummy variable indicating whether the decision is a reasoned opinion sent by the Commission under Article 258 of the Treaty on the Functioning of the European Union (TFEU).
\item[\code{stage\_rf\_258}] \code{dummy}\hspace{5pt}A dummy variable indicating whether the decision is a referral to the Court of Justice of the European Union (CJEU) made by the Commission under Article 258 of the Treaty on the Functioning of the European Union (TFEU).
\item[\code{stage\_lfn\_260}] \code{dummy}\hspace{5pt}A dummy variable indicating whether the decision is a letter of formal notice sent by the Commission under Article 260 of the Treaty on the Functioning of the European Union (TFEU).
\item[\code{stage\_ro\_260}] \code{dummy}\hspace{5pt}A dummy variable indicating whether the decision is a reasoned opinion sent by the Commission under Article 260 of the Treaty on the Functioning of the European Union (TFEU).
\item[\code{stage\_rf\_260}] \code{dummy}\hspace{5pt}A dummy variable indicating whether the decision is a referral to the Court of Justice of the European Union (CJEU) made by the Commission under Article 260 of the Treaty on the Functioning of the European Union (TFEU).
\item[\code{stage\_closing}] \code{dummy}\hspace{5pt}A dummy variable indicating whether the decision is the closing of the case by the Commission. 
\item[\code{stage\_withdrawal}] \code{dummy}\hspace{5pt}A dummy variable indicating whether the decision is the withdrawal of the case by the Commission.
\item[\code{case\_history}] \code{string}\hspace{5pt}A summary of the history of the case, including all decisions taken by the Commission in the case (with dates). 
\item[\code{corrected\_lfn\_258}] \code{dummy}\hspace{5pt}A dummy variable indicating whether the decision is a letter of formal notice sent by the Commission under Article 258 of the Treaty on the Functioning of the European Union (TFEU).
\item[\code{corrected\_ro\_258}] \code{dummy}\hspace{5pt}A dummy variable indicating whether the decision is a reasoned opinion sent by the Commission under Article 258 of the Treaty on the Functioning of the European Union (TFEU).
\item[\code{corrected\_rf\_258}] \code{dummy}\hspace{5pt}A dummy variable indicating whether the decision is a referral to the Court of Justice of the European Union (CJEU) made by the Commission under Article 258 of the Treaty on the Functioning of the European Union (TFEU).
\item[\code{corrected\_lfn\_260}] \code{dummy}\hspace{5pt}A dummy variable indicating whether the decision is a letter of formal notice sent by the Commission under Article 260 of the Treaty on the Functioning of the European Union (TFEU).
\item[\code{corrected\_ro\_260}] \code{dummy}\hspace{5pt}A dummy variable indicating whether the decision is a reasoned opinion sent by the Commission under Article 260 of the Treaty on the Functioning of the European Union (TFEU).
\item[\code{corrected\_rf\_260}] \code{dummy}\hspace{5pt}A dummy variable indicating whether the decision is a referral to the Court of Justice of the European Union (CJEU) made by the Commission under Article 260 of the Treaty on the Functioning of the European Union (TFEU). Corrected based on data about decisions made by the Commission at later stages of the infringement process.
\item[\code{count\_decisions}] \code{numeric}\hspace{5pt}The number of decisions issued by the Commission in the case. Based on the corrected data.
\item[\code{count\_lfn\_258}] \code{numeric}\hspace{5pt}The number of letters of formal notice sent by the Commission under Article 258 of the Treaty on the Functioning of the European Union (TFEU).
\item[\code{count\_ro\_258}] \code{numeric}\hspace{5pt}The number of reasoned opinions sent by the Commission under Article 258 of the Treaty on the Functioning of the European Union (TFEU).
\item[\code{count\_rf\_258}] \code{numeric}\hspace{5pt}The number of referrals to the Court of Justice of the European Union (CJEU) made by the Commission under Article 258 of the Treaty on the Functioning of the European Union (TFEU).
\item[\code{count\_lfn\_260}] \code{numeric}\hspace{5pt}The number of letters of formal notice sent by the Commission under Article 260 of the Treaty on the Functioning of the European Union (TFEU).
\item[\code{count\_ro\_260}] \code{numeric}\hspace{5pt}The number of reasoned opinions sent by the Commission under Article 260 of the Treaty on the Functioning of the European Union (TFEU).
\item[\code{count\_rf\_260}] \code{numeric}\hspace{5pt}The number of referrals to the Court of Justice of the European Union (CJEU) made by the Commission under Article 260 of the Treaty on the Functioning of the European Union (TFEU).
\item[\code{count\_missing}] \code{numeric}\hspace{5pt}The number of decision stages prior to the last observed decision stage that are missing.
\item[\code{missing\_lfn\_258}] \code{dummy}\hspace{5pt}A dummy variable indicating whether the Commission does not report a letter of formal notice under Article 258 of the Treaty on the Functioning of the European Union (TFEU) but does report a decision at a later stage of the infringement process.
\item[\code{missing\_ro\_258}] \code{dummy}\hspace{5pt}A dummy variable indicating whether the Commission does not report a reasoned opinion under Article 258 of the Treaty on the Functioning of the European Union (TFEU) but does report a decision at a later stage of the infringement process.
\item[\code{missing\_rf\_258}] \code{dummy}\hspace{5pt}A dummy variable indicating whether the Commission does not report a referral to the Court of Justice of the European Union (CJEU) under Article 258 of the Treaty on the Functioning of the European Union (TFEU) but does report a decision at a later stage of the infringement process.
\item[\code{missing\_lfn\_260}] \code{dummy}\hspace{5pt}A dummy variable indicating whether the Commission does not report a letter of formal notice under Article 260 of the Treaty on the Functioning of the European Union (TFEU) but does report a decision at a later stage of the infringement process.
\item[\code{missing\_ro\_260}] \code{dummy}\hspace{5pt}A dummy variable indicating whether the Commission does not report a reasoned opinion under Article 260 of the Treaty on the Functioning of the European Union (TFEU) but does report a decision at a later stage of the infringement process.
\end{description}
%--------------------------------------------------%
% dataset
%--------------------------------------------------%

\headerpage{cases\_ts}{Case-level time-series data}{30}{12}

\subheading{Description}

This dataset includes aggregated data on the number of infringement cases per year (time-series data). There is one observation per year (2002-2020).

\subheading{Variables}

\begin{description}[labelwidth=130pt, leftmargin=\dimexpr\labelwidth+\labelsep\relax, font=\normalfont, itemsep=10pt]
\item[\code{key\_id}] \code{numeric}\hspace{5pt}An ID number that uniquely identifies each observation in the dataset. 
\item[\code{year}] \code{numeric}\hspace{5pt}The year the case was opened by the Commission.
\item[\code{count\_cases}] \code{numeric}\hspace{5pt}A count of the number of cases opened by the Commission at this level of aggregation.
\end{description}
%--------------------------------------------------%
% dataset
%--------------------------------------------------%

\headerpage{cases\_ts\_ct}{Case-level time-series data by case type}{30}{12}

\subheading{Description}

This dataset includes aggregated data on the number of infringement cases per year (time-series data) broken down by case type (noncommunication vs nonconformity). There is one observation per year per case type (2002-2020).

\subheading{Variables}

\begin{description}[labelwidth=130pt, leftmargin=\dimexpr\labelwidth+\labelsep\relax, font=\normalfont, itemsep=10pt]
\item[\code{key\_id}] \code{numeric}\hspace{5pt}An ID number that uniquely identifies each observation in the dataset. 
\item[\code{year}] \code{numeric}\hspace{5pt}The year the case was opened by the Commission.
\item[\code{case\_type\_id}] \code{numeric}\hspace{5pt}An ID number that uniquely identifies each type of state aid cases. Coded \code{1} for noncommunication cases, which are cases that relate to a member state failing to notify the Commission that it has transposed a directive by the stated deadline. Coded \code{2} for nonconformity cases, which are cases that relate to a member state incorrectly transposing a directive. 
\item[\code{case\_type}] \code{string}\hspace{5pt}The type of the infringement case. There are two types of cases. Coded \code{noncommunication} for cases that relate to a member state failing to notify the Commission that it has transposed a directive by the stated deadline. Coded \code{nonconformity} for cases that relate to a member state incorrectly transposing a directive. 
\item[\code{count\_cases}] \code{numeric}\hspace{5pt}A count of the number of cases opened by the Commission at this level of aggregation.
\end{description}
%--------------------------------------------------%
% dataset
%--------------------------------------------------%

\headerpage{cases\_csts\_ms}{Case-level cross-sectional time-series data by member state}{30}{12}

\subheading{Description}

This dataset includes aggregated data on the number of infringement cases per member state per year (cross-sectional time-series data). There is one observation per member state per year (2002-2020), excluding state-years where the state was not a member of the EU.

\subheading{Variables}

\begin{description}[labelwidth=130pt, leftmargin=\dimexpr\labelwidth+\labelsep\relax, font=\normalfont, itemsep=10pt]
\item[\code{key\_id}] \code{numeric}\hspace{5pt}An ID number that uniquely identifies each observation in the dataset. 
\item[\code{year}] \code{numeric}\hspace{5pt}The year the case was opened by the Commission.
\item[\code{member\_state\_id}] \code{numeric}\hspace{5pt}An ID number that uniquely identifies each member state. This ID number is assigned when member states are sorted by accession date and then alphabetically. 
\item[\code{member\_state}] \code{string}\hspace{5pt}The name of the member state that the Commission opened the case against. 
\item[\code{member\_state\_code}] \code{string}\hspace{5pt}A two letter code assigned by the Commission that uniquely identifies each member state. 
\item[\code{count\_cases}] \code{numeric}\hspace{5pt}A count of the number of cases opened by the Commission at this level of aggregation.
\end{description}
%--------------------------------------------------%
% dataset
%--------------------------------------------------%

\headerpage{cases\_csts\_ms\_ct}{Case-level cross-sectional time-series data by member state and case type}{30}{12}

\subheading{Description}

This dataset includes aggregated data on the number of infringement cases per member state per year (cross-sectional time-series data) broken down by case type (noncommunication vs nonconformity). There is one observation per member state per year per case type (2002-2020), excluding state-years where the state was not a member of the EU.

\subheading{Variables}

\begin{description}[labelwidth=130pt, leftmargin=\dimexpr\labelwidth+\labelsep\relax, font=\normalfont, itemsep=10pt]
\item[\code{key\_id}] \code{numeric}\hspace{5pt}An ID number that uniquely identifies each observation in the dataset. 
\item[\code{year}] \code{numeric}\hspace{5pt}The year the case was opened by the Commission.
\item[\code{member\_state\_id}] \code{numeric}\hspace{5pt}An ID number that uniquely identifies each member state. This ID number is assigned when member states are sorted by accession date and then alphabetically. 
\item[\code{member\_state}] \code{string}\hspace{5pt}The name of the member state that the Commission opened the case against. 
\item[\code{member\_state\_code}] \code{string}\hspace{5pt}A two letter code assigned by the Commission that uniquely identifies each member state. 
\item[\code{case\_type\_id}] \code{numeric}\hspace{5pt}An ID number that uniquely identifies each type of state aid cases. Coded \code{1} for noncommunication cases, which are cases that relate to a member state failing to notify the Commission that it has transposed a directive by the stated deadline. Coded \code{2} for nonconformity cases, which are cases that relate to a member state incorrectly transposing a directive. 
\item[\code{case\_type}] \code{string}\hspace{5pt}The type of the infringement case. There are two types of cases. Coded \code{noncommunication} for cases that relate to a member state failing to notify the Commission that it has transposed a directive by the stated deadline. Coded \code{nonconformity} for cases that relate to a member state incorrectly transposing a directive. 
\item[\code{count\_cases}] \code{numeric}\hspace{5pt}A count of the number of cases opened by the Commission at this level of aggregation.
\end{description}
%--------------------------------------------------%
% dataset
%--------------------------------------------------%

\headerpage{cases\_csts\_dp}{Case-level cross-sectional time-series data by department}{30}{12}

\subheading{Description}

This dataset includes aggregated data on the number of infringement cases per department per year (cross-sectional time-series data). There is one observation per department per year (2002-2020).

\subheading{Variables}

\begin{description}[labelwidth=130pt, leftmargin=\dimexpr\labelwidth+\labelsep\relax, font=\normalfont, itemsep=10pt]
\item[\code{key\_id}] \code{numeric}\hspace{5pt}An ID number that uniquely identifies each observation in the dataset. 
\item[\code{year}] \code{numeric}\hspace{5pt}The year the case was opened by the Commission.
\item[\code{department\_id}] \code{numeric}\hspace{5pt}An ID number that uniquely identifies each Commission department.
\item[\code{department}] \code{string}\hspace{5pt}The name of the Commission department that opened the infringement case.
\item[\code{department\_code}] \code{string}\hspace{5pt}A multi-letter code assigned by the Commission that uniquely identifies each department.
\item[\code{count\_cases}] \code{numeric}\hspace{5pt}A count of the number of cases opened by the Commission at this level of aggregation.
\end{description}
%--------------------------------------------------%
% dataset
%--------------------------------------------------%

\headerpage{cases\_csts\_dp\_ct}{Case-level cross-sectional time-series data by department and case type}{30}{12}

\subheading{Description}

This dataset includes aggregated data on the number of infringement cases per department per year (cross-sectional time-series data) broken down by case type (noncommunication vs nonconformity). There is one observation per department per year per case type (2002-2020). The dataset uses current department names. 

\subheading{Variables}

\begin{description}[labelwidth=130pt, leftmargin=\dimexpr\labelwidth+\labelsep\relax, font=\normalfont, itemsep=10pt]
\item[\code{key\_id}] \code{numeric}\hspace{5pt}An ID number that uniquely identifies each observation in the dataset. 
\item[\code{year}] \code{numeric}\hspace{5pt}The year the case was opened by the Commission.
\item[\code{department\_id}] \code{numeric}\hspace{5pt}An ID number that uniquely identifies each Commission department.
\item[\code{department}] \code{string}\hspace{5pt}The name of the Commission department that opened the infringement case.
\item[\code{department\_code}] \code{string}\hspace{5pt}A multi-letter code assigned by the Commission that uniquely identifies each department.
\item[\code{case\_type\_id}] \code{numeric}\hspace{5pt}An ID number that uniquely identifies each type of state aid cases. Coded \code{1} for noncommunication cases, which are cases that relate to a member state failing to notify the Commission that it has transposed a directive by the stated deadline. Coded \code{2} for nonconformity cases, which are cases that relate to a member state incorrectly transposing a directive. 
\item[\code{case\_type}] \code{string}\hspace{5pt}The type of the infringement case. There are two types of cases. Coded \code{noncommunication} for cases that relate to a member state failing to notify the Commission that it has transposed a directive by the stated deadline. Coded \code{nonconformity} for cases that relate to a member state incorrectly transposing a directive. 
\item[\code{count\_cases}] \code{numeric}\hspace{5pt}A count of the number of cases opened by the Commission at this level of aggregation.
\end{description}
%--------------------------------------------------%
% dataset
%--------------------------------------------------%

\headerpage{cases\_ddy}{Case-level directed dyad-year data}{30}{12}

\subheading{Description}

This dataset includes aggregated data on the number of infringement cases per department per member state per year (directed dyad-year data). There is one observation per department per member state per year (2002-2020), excluding directed dyad-years where the state was not a member of the EU. The dataset uses current department names. 

\subheading{Variables}

\begin{description}[labelwidth=130pt, leftmargin=\dimexpr\labelwidth+\labelsep\relax, font=\normalfont, itemsep=10pt]
\item[\code{key\_id}] \code{numeric}\hspace{5pt}An ID number that uniquely identifies each observation in the dataset. 
\item[\code{year}] \code{numeric}\hspace{5pt}The year the case was opened by the Commission.
\item[\code{department\_id}] \code{numeric}\hspace{5pt}An ID number that uniquely identifies each Commission department.
\item[\code{department}] \code{string}\hspace{5pt}The name of the Commission department that opened the infringement case.
\item[\code{department\_code}] \code{string}\hspace{5pt}A multi-letter code assigned by the Commission that uniquely identifies each department.
\item[\code{member\_state\_id}] \code{numeric}\hspace{5pt}An ID number that uniquely identifies each member state. This ID number is assigned when member states are sorted by accession date and then alphabetically. 
\item[\code{member\_state}] \code{string}\hspace{5pt}The name of the member state that the Commission opened the case against. 
\item[\code{member\_state\_code}] \code{string}\hspace{5pt}A two letter code assigned by the Commission that uniquely identifies each member state. 
\item[\code{count\_cases}] \code{numeric}\hspace{5pt}A count of the number of cases opened by the Commission at this level of aggregation.
\end{description}
%--------------------------------------------------%
% dataset
%--------------------------------------------------%

\headerpage{cases\_ddy\_ct}{Case-level directed dyad-year data by case type}{30}{12}

\subheading{Description}

This dataset includes aggregated data on the number of infringement cases per department per member state per year (directed dyad-year data) broken down by case type (noncommunication vs nonconformity). There is one observation per department per member state per year per case type (2002-2020), excluding directed dyad-years where the state was not a member of the EU. The dataset uses current department names. 

\subheading{Variables}

\begin{description}[labelwidth=130pt, leftmargin=\dimexpr\labelwidth+\labelsep\relax, font=\normalfont, itemsep=10pt]
\item[\code{key\_id}] \code{numeric}\hspace{5pt}An ID number that uniquely identifies each observation in the dataset. 
\item[\code{year}] \code{numeric}\hspace{5pt}The year the case was opened by the Commission.
\item[\code{department\_id}] \code{numeric}\hspace{5pt}An ID number that uniquely identifies each Commission department.
\item[\code{department}] \code{string}\hspace{5pt}The name of the Commission department that opened the infringement case.
\item[\code{department\_code}] \code{string}\hspace{5pt}A multi-letter code assigned by the Commission that uniquely identifies each department.
\item[\code{member\_state\_id}] \code{numeric}\hspace{5pt}An ID number that uniquely identifies each member state. This ID number is assigned when member states are sorted by accession date and then alphabetically. 
\item[\code{member\_state}] \code{string}\hspace{5pt}The name of the member state that the Commission opened the case against. 
\item[\code{member\_state\_code}] \code{string}\hspace{5pt}A two letter code assigned by the Commission that uniquely identifies each member state. 
\item[\code{case\_type\_id}] \code{numeric}\hspace{5pt}An ID number that uniquely identifies each type of state aid cases. Coded \code{1} for noncommunication cases, which are cases that relate to a member state failing to notify the Commission that it has transposed a directive by the stated deadline. Coded \code{2} for nonconformity cases, which are cases that relate to a member state incorrectly transposing a directive. 
\item[\code{case\_type}] \code{string}\hspace{5pt}The type of the infringement case. There are two types of cases. Coded \code{noncommunication} for cases that relate to a member state failing to notify the Commission that it has transposed a directive by the stated deadline. Coded \code{nonconformity} for cases that relate to a member state incorrectly transposing a directive. 
\item[\code{count\_cases}] \code{numeric}\hspace{5pt}A count of the number of cases opened by the Commission at this level of aggregation.
\end{description}
%--------------------------------------------------%
% dataset
%--------------------------------------------------%

\headerpage{cases\_net}{Case-level network data}{30}{12}

\subheading{Description}

This dataset includes network data on infringement cases. Network data is similar to directed dyad-year data except that it only includes directed dyad-years with at least one infringement case. For every year, there is one node per department and one node per member state. Edges can only exist between a department and a member state. There is an edge between a department and a member state if and only if the department opened at least one case against the member state during that year. The weight of the edge is the number of cases that the department opened against the member state. There is one observation per department per member state per year (2002-2020), excluding directed dyad-years where the state was not a member of the EU, but only if count of cases is positive.

\subheading{Variables}

\begin{description}[labelwidth=130pt, leftmargin=\dimexpr\labelwidth+\labelsep\relax, font=\normalfont, itemsep=10pt]
\item[\code{key\_id}] \code{numeric}\hspace{5pt}An ID number that uniquely identifies each observation in the dataset. 
\item[\code{year}] \code{numeric}\hspace{5pt}The year the case was opened by the Commission.
\item[\code{from\_node\_id}] \code{numeric}\hspace{5pt}An ID number that uniquely identifies each node in the network that creates a link, which is always a Commission department.
\item[\code{from\_node}] \code{string}\hspace{5pt}The name of the Commission department that opened the infringement case.
\item[\code{to\_node\_id}] \code{numeric}\hspace{5pt}An ID number that uniquely identifies each node in the network that receives a link, which is always a member state.
\item[\code{to\_node}] \code{string}\hspace{5pt}The name of the member state that the Commission opened the case against. 
\item[\code{edge\_weight}] \code{numeric}\hspace{5pt}The weight of the edge, which is the number of cases opened by the Commission.
\end{description}
%--------------------------------------------------%
% dataset
%--------------------------------------------------%

\headerpage{cases\_net\_ct}{Case-level network data by case type}{30}{12}

\subheading{Description}

This dataset includes multi-dimensional network data on infringement cases. There is one dimension per case type (noncommunication vs nonconformity). Network data is similar to directed dyad-year data except that it only includes directed dyad-years with at least one infringement case with respect to each case type. For every year, there is one node per department and one node per member state. Edges can only exist between a department and a member state. There is an edge between a department and a member state with respect to each case type if and only if the department opened at least one case against the member state during that year. The weight of the edge is the number of cases that the department opened against the member state. There is one observation per department per member state per year per case type (2002-2020), excluding directed dyad-years where the state was not a member of the EU, but only if the count of cases is positive.

\subheading{Variables}

\begin{description}[labelwidth=130pt, leftmargin=\dimexpr\labelwidth+\labelsep\relax, font=\normalfont, itemsep=10pt]
\item[\code{key\_id}] \code{numeric}\hspace{5pt}An ID number that uniquely identifies each observation in the dataset. 
\item[\code{year}] \code{numeric}\hspace{5pt}The year the case was opened by the Commission.
\item[\code{layer\_id}] \code{numeric}\hspace{5pt}An ID number that uniquely identifies each layer of the network.
\item[\code{layer}] \code{string}\hspace{5pt}The layer of the network, which is type of case.
\item[\code{from\_node\_id}] \code{numeric}\hspace{5pt}An ID number that uniquely identifies each node in the network that creates a link, which is always a Commission department.
\item[\code{from\_node}] \code{string}\hspace{5pt}The name of the Commission department that opened the infringement case.
\item[\code{to\_node\_id}] \code{numeric}\hspace{5pt}An ID number that uniquely identifies each node in the network that receives a link, which is always a member state.
\item[\code{to\_node}] \code{string}\hspace{5pt}The name of the member state that the Commission opened the case against. 
\item[\code{edge\_weight}] \code{numeric}\hspace{5pt}The weight of the edge, which is the number of cases opened by the Commission.
\end{description}
%--------------------------------------------------%
% dataset
%--------------------------------------------------%

\headerpage{decisions}{Decision-level data}{30}{12}

\subheading{Description}

This dataset include data on infringement decisions. There is one observation per decision per case (2002-2020). The dataset includes information on the department responsible for the decision and the member state that the decision is against.

\subheading{Variables}

\begin{description}[labelwidth=130pt, leftmargin=\dimexpr\labelwidth+\labelsep\relax, font=\normalfont, itemsep=10pt]
\item[\code{key\_id}] \code{numeric}\hspace{5pt}An ID number that uniquely identifies each observation in the dataset. 
\item[\code{decision\_id}] \code{numeric}\hspace{5pt}An ID number that uniquely identifies each decision in an infringement case.
\item[\code{case\_id}] \code{string}\hspace{5pt}An ID number that uniquely identifies each case.
\item[\code{case\_number}] \code{numeric}\hspace{5pt}The number of the case.
\item[\code{case\_year}] \code{numeric}\hspace{5pt}The year of the case.
\item[\code{decision\_date}] \code{date}\hspace{5pt}The date that the Commission issued the decision in the format \code{YYYY-MM-DD}.
\item[\code{decision\_year}] \code{numeric}\hspace{5pt}The year that the Commission issued the decision.
\item[\code{decision\_month}] \code{numeric}\hspace{5pt}The month that the Commission issued the decision.
\item[\code{decision\_day}] \code{numeric}\hspace{5pt}The day that the Commission issued the decision.
\item[\code{member\_state\_id}] \code{numeric}\hspace{5pt}An ID number that uniquely identifies each member state. This ID number is assigned when member states are sorted by accession date and then alphabetically. 
\item[\code{member\_state}] \code{string}\hspace{5pt}The name of the member state that the Commission opened the case against. 
\item[\code{member\_state\_code}] \code{string}\hspace{5pt}A two letter code assigned by the Commission that uniquely identifies each member state. 
\item[\code{department\_id}] \code{numeric}\hspace{5pt}An ID number that uniquely identifies each Commission department.
\item[\code{department}] \code{string}\hspace{5pt}The name of the Commission department that opened the infringement case.
\item[\code{department\_code}] \code{string}\hspace{5pt}A multi-letter code assigned by the Commission that uniquely identifies each department.
\item[\code{case\_type\_id}] \code{numeric}\hspace{5pt}An ID number that uniquely identifies each type of state aid cases. Coded \code{1} for noncommunication cases, which are cases that relate to a member state failing to notify the Commission that it has transposed a directive by the stated deadline. Coded \code{2} for nonconformity cases, which are cases that relate to a member state incorrectly transposing a directive. 
\item[\code{case\_type}] \code{string}\hspace{5pt}The type of the infringement case. There are two types of cases. Coded \code{noncommunication} for cases that relate to a member state failing to notify the Commission that it has transposed a directive by the stated deadline. Coded \code{nonconformity} for cases that relate to a member state incorrectly transposing a directive. 
\item[\code{noncommunication}] \code{dummy}\hspace{5pt}A dummy variable indicating whether the infringement case is a noncommunication case, which is a case that relates to a member state failing to notify that Commission that it has transposed a directive by the stated deadline.
\item[\code{nonconformity}] \code{dummy}\hspace{5pt}A dummy variable indicating whether the infringement case is a nonconformity case, which is a case that relates to a member state incorrectly transposing a directive.
\item[\code{directive}] \code{dummy}\hspace{5pt}A dummy variable indicating whether the description of the infringement case provided by the Commission references a directive that the case relates to. Directive numbers have the format \code{\#\#\#\#/\#\#\#\#}, where the first set of numbers is the year the directive was enacted and the second set is the number of the directive.
\item[\code{directive\_number}] \code{string}\hspace{5pt}The number of the directive that the infringement case relates to, if applicable. Coded \code{NA} if the description of the case provided by the Commission does not reference a specific directive number.
\item[\code{celex}] \code{string}\hspace{5pt}The CELEX number of the directive that the infringement case relates to, if applicable. Coded \code{NA} if the description of the case provided by the Commission does not reference a specific directive number. CELEX numbers for directives have the format \code{3\#\#\#\#L\#\#\#\#}, where the first set of numbers is the year the directive was enacted and the second set is the number of the directive.
\item[\code{decision\_stage\_id}] \code{numeric}\hspace{5pt}An ID number that uniquely identifies each decision stage in the infringement procedure. Coded \code{1} for letters of formal notice under Article 258 of the Treaty on the Functioning of the European Union (TFEU), coded \code{2} for reasoned opinions under Article 258, coded \code{3} for referrals to the Court of Justice of the European Union (CJEU) under Article 258, coded \code{4} for letters of formal notice under Article 260, coded \code{5} for reasoned opinions under Article 260, and coded \code{6} for referrals to the Court under Article 260.
\item[\code{decision\_stage}] \code{string}\hspace{5pt}The decision stage of the infringement procedure. Possible values include: \code{Letter of formal notice (Article 258)}, \code{Reasoned opinion (Article 259)}, \code{Referral to the Court (Article 258)}, \code{Letter of formal notice (Article 260)}, \code{Reasoned opinion (Article 260)}, and \code{Referral to the Court (Article 260)}.
\item[\code{stage\_lfn\_258}] \code{dummy}\hspace{5pt}A dummy variable indicating whether the decision is a letter of formal notice sent by the Commission under Article 258 of the Treaty on the Functioning of the European Union (TFEU).
\item[\code{stage\_ro\_258}] \code{dummy}\hspace{5pt}A dummy variable indicating whether the decision is a reasoned opinion sent by the Commission under Article 258 of the Treaty on the Functioning of the European Union (TFEU).
\item[\code{stage\_rf\_258}] \code{dummy}\hspace{5pt}A dummy variable indicating whether the decision is a referral to the Court of Justice of the European Union (CJEU) made by the Commission under Article 258 of the Treaty on the Functioning of the European Union (TFEU).
\item[\code{stage\_lfn\_260}] \code{dummy}\hspace{5pt}A dummy variable indicating whether the decision is a letter of formal notice sent by the Commission under Article 260 of the Treaty on the Functioning of the European Union (TFEU).
\item[\code{stage\_ro\_260}] \code{dummy}\hspace{5pt}A dummy variable indicating whether the decision is a reasoned opinion sent by the Commission under Article 260 of the Treaty on the Functioning of the European Union (TFEU).
\item[\code{stage\_rf\_260}] \code{dummy}\hspace{5pt}A dummy variable indicating whether the decision is a referral to the Court of Justice of the European Union (CJEU) made by the Commission under Article 260 of the Treaty on the Functioning of the European Union (TFEU).
\item[\code{stage\_closing}] \code{dummy}\hspace{5pt}A dummy variable indicating whether the decision is the closing of the case by the Commission. 
\item[\code{stage\_withdrawal}] \code{dummy}\hspace{5pt}A dummy variable indicating whether the decision is the withdrawal of the case by the Commission.
\item[\code{stage\_additional}] \code{dummy}\hspace{5pt}A dummy variable indicating whether the decision is an additional decision (a decision of the same type as a previous decision). 
\item[\code{press\_release}] \code{dummy}\hspace{5pt}A dummy variable indicating whether the Commission published a press release to publicize the decision.
\end{description}
%--------------------------------------------------%
% dataset
%--------------------------------------------------%

\headerpage{decisions\_ts}{Decision-level time-series data}{30}{12}

\subheading{Description}

This dataset includes aggregated data on the number of decisions per stage of the infringement procedure per year (time-series data). There is one observation per year per decision stage (2002-2020).

\subheading{Variables}

\begin{description}[labelwidth=130pt, leftmargin=\dimexpr\labelwidth+\labelsep\relax, font=\normalfont, itemsep=10pt]
\item[\code{key\_id}] \code{numeric}\hspace{5pt}An ID number that uniquely identifies each observation in the dataset. 
\item[\code{year}] \code{numeric}\hspace{5pt}The year the decision was issued by the Commission.
\item[\code{decision\_stage\_id}] \code{numeric}\hspace{5pt}An ID number that uniquely identifies each decision stage in the infringement procedure. Coded \code{1} for letters of formal notice under Article 258 of the Treaty on the Functioning of the European Union (TFEU), coded \code{2} for reasoned opinions under Article 258, coded \code{3} for referrals to the Court under Article 258, coded \code{4} for letters of formal notice under Article 260, coded \code{5} for reasoned opinions under Article 260, and coded \code{6} for referrals to the Court under Article 261
\item[\code{decision\_stage}] \code{string}\hspace{5pt}The decision stage of the infringement procedure. Possible values include: \code{Letter of formal notice (Article 258)}, \code{Reasoned opinion (Article 259)}, \code{Referral to the Court (Article 258)}, \code{Letter of formal notice (Article 260)}, \code{Reasoned opinion (Article 260)}, and \code{Referral to the Court (Article 260)}. 
\item[\code{count\_decisions}] \code{numeric}\hspace{5pt}A count of the number of decisions made by the Commission in infringement cases at this level of aggregation.
\end{description}
%--------------------------------------------------%
% dataset
%--------------------------------------------------%

\headerpage{decisions\_ts\_ct}{Decision-level time-series data by case type}{30}{12}

\subheading{Description}

This dataset includes aggregated data on the number of decisions per stage of the infringement procedure per year (time-series data) broken down by case type (nonommunication vs nonconformity). There is one observation per year per decision stage per case type (2002-2020).

\subheading{Variables}

\begin{description}[labelwidth=130pt, leftmargin=\dimexpr\labelwidth+\labelsep\relax, font=\normalfont, itemsep=10pt]
\item[\code{key\_id}] \code{numeric}\hspace{5pt}An ID number that uniquely identifies each observation in the dataset. 
\item[\code{year}] \code{numeric}\hspace{5pt}The year the decision was issued by the Commission.
\item[\code{case\_type\_id}] \code{numeric}\hspace{5pt}An ID number that uniquely identifies each type of state aid cases. Coded \code{1} for noncommunication cases, which are cases that relate to a member state failing to notify the Commission that it has transposed a directive by the stated deadline. Coded \code{2} for nonconformity cases, which are cases that relate to a member state incorrectly transposing a directive. 
\item[\code{case\_type}] \code{string}\hspace{5pt}The type of the infringement case. There are two types of cases. Coded \code{noncommunication} for cases that relate to a member state failing to notify the Commission that it has transposed a directive by the stated deadline. Coded \code{nonconformity} for cases that relate to a member state incorrectly transposing a directive. 
\item[\code{decision\_stage\_id}] \code{numeric}\hspace{5pt}An ID number that uniquely identifies each decision stage in the infringement procedure. Coded \code{1} for letters of formal notice under Article 258 of the Treaty on the Functioning of the European Union (TFEU), coded \code{2} for reasoned opinions under Article 258, coded \code{3} for referrals to the Court under Article 258, coded \code{4} for letters of formal notice under Article 260, coded \code{5} for reasoned opinions under Article 260, and coded \code{6} for referrals to the Court under Article 262
\item[\code{decision\_stage}] \code{string}\hspace{5pt}The decision stage of the infringement procedure. Possible values include: \code{Letter of formal notice (Article 258)}, \code{Reasoned opinion (Article 259)}, \code{Referral to the Court (Article 258)}, \code{Letter of formal notice (Article 260)}, \code{Reasoned opinion (Article 260)}, and \code{Referral to the Court (Article 260)}. 
\item[\code{count\_cases}] \code{numeric}\hspace{5pt}A count of the number of cases opened by the Commission at this level of aggregation.
\end{description}
%--------------------------------------------------%
% dataset
%--------------------------------------------------%

\headerpage{decisions\_csts\_ms}{Decision-level cross-sectional time-series data by member state}{30}{12}

\subheading{Description}

This dataset includes aggregated data on the number of decisions per stage of the infringement procedure per member state per year (cross-sectional time-series data). There is one observation per member state per year per decision stage (2002-2020), excluding state-years where the state was not a member of the EU.

\subheading{Variables}

\begin{description}[labelwidth=130pt, leftmargin=\dimexpr\labelwidth+\labelsep\relax, font=\normalfont, itemsep=10pt]
\item[\code{key\_id}] \code{numeric}\hspace{5pt}An ID number that uniquely identifies each observation in the dataset. 
\item[\code{year}] \code{numeric}\hspace{5pt}The year the decision was issued by the Commission.
\item[\code{member\_state\_id}] \code{numeric}\hspace{5pt}An ID number that uniquely identifies each member state. This ID number is assigned when member states are sorted by accession date and then alphabetically. 
\item[\code{member\_state}] \code{string}\hspace{5pt}The name of the member state that the Commission opened the case against. 
\item[\code{member\_state\_code}] \code{string}\hspace{5pt}A two letter code assigned by the Commission that uniquely identifies each member state. 
\item[\code{decision\_stage\_id}] \code{numeric}\hspace{5pt}An ID number that uniquely identifies each decision stage in the infringement procedure. Coded \code{1} for letters of formal notice under Article 258 of the Treaty on the Functioning of the European Union (TFEU), coded \code{2} for reasoned opinions under Article 258, coded \code{3} for referrals to the Court under Article 258, coded \code{4} for letters of formal notice under Article 260, coded \code{5} for reasoned opinions under Article 260, and coded \code{6} for referrals to the Court under Article 263
\item[\code{decision\_stage}] \code{string}\hspace{5pt}The decision stage of the infringement procedure. Possible values include: \code{Letter of formal notice (Article 258)}, \code{Reasoned opinion (Article 259)}, \code{Referral to the Court (Article 258)}, \code{Letter of formal notice (Article 260)}, \code{Reasoned opinion (Article 260)}, and \code{Referral to the Court (Article 260)}. 
\item[\code{count\_decisions}] \code{numeric}\hspace{5pt}A count of the number of decisions made by the Commission in infringement cases at this level of aggregation.
\end{description}
%--------------------------------------------------%
% dataset
%--------------------------------------------------%

\headerpage{decisions\_csts\_ms\_ct}{Decision-level cross-sectional time-series data by member state and case type}{30}{12}

\subheading{Description}

This dataset includes aggregated data on the number of decisions per stage of the infringement procedure per member state per year (cross-sectional time-series data) broken down by case type (noncommunication vs nonconformity). There is one observation per member state per year per decision stage per case type (2002-2020), excluding state-years where the state was not a member of the EU.

\subheading{Variables}

\begin{description}[labelwidth=130pt, leftmargin=\dimexpr\labelwidth+\labelsep\relax, font=\normalfont, itemsep=10pt]
\item[\code{key\_id}] \code{numeric}\hspace{5pt}An ID number that uniquely identifies each observation in the dataset. 
\item[\code{year}] \code{numeric}\hspace{5pt}The year the decision was issued by the Commission.
\item[\code{member\_state\_id}] \code{numeric}\hspace{5pt}An ID number that uniquely identifies each member state. This ID number is assigned when member states are sorted by accession date and then alphabetically. 
\item[\code{member\_state}] \code{string}\hspace{5pt}The name of the member state that the Commission opened the case against. 
\item[\code{member\_state\_code}] \code{string}\hspace{5pt}A two letter code assigned by the Commission that uniquely identifies each member state. 
\item[\code{case\_type\_id}] \code{numeric}\hspace{5pt}An ID number that uniquely identifies each type of state aid cases. Coded \code{1} for noncommunication cases, which are cases that relate to a member state failing to notify the Commission that it has transposed a directive by the stated deadline. Coded \code{2} for nonconformity cases, which are cases that relate to a member state incorrectly transposing a directive. 
\item[\code{case\_type}] \code{string}\hspace{5pt}The type of the infringement case. There are two types of cases. Coded \code{noncommunication} for cases that relate to a member state failing to notify the Commission that it has transposed a directive by the stated deadline. Coded \code{nonconformity} for cases that relate to a member state incorrectly transposing a directive. 
\item[\code{decision\_stage\_id}] \code{numeric}\hspace{5pt}An ID number that uniquely identifies each decision stage in the infringement procedure. Coded \code{1} for letters of formal notice under Article 258 of the Treaty on the Functioning of the European Union (TFEU), coded \code{2} for reasoned opinions under Article 258, coded \code{3} for referrals to the Court under Article 258, coded \code{4} for letters of formal notice under Article 260, coded \code{5} for reasoned opinions under Article 260, and coded \code{6} for referrals to the Court under Article 264
\item[\code{decision\_stage}] \code{string}\hspace{5pt}The decision stage of the infringement procedure. Possible values include: \code{Letter of formal notice (Article 258)}, \code{Reasoned opinion (Article 259)}, \code{Referral to the Court (Article 258)}, \code{Letter of formal notice (Article 260)}, \code{Reasoned opinion (Article 260)}, and \code{Referral to the Court (Article 260)}. 
\item[\code{count\_decisions}] \code{numeric}\hspace{5pt}A count of the number of decisions made by the Commission in infringement cases at this level of aggregation.
\end{description}
%--------------------------------------------------%
% dataset
%--------------------------------------------------%

\headerpage{decisions\_csts\_dp}{Decision-level cross-sectional time-series data by department}{30}{12}

\subheading{Description}

This dataset includes aggregated data on the number of decisions per stage of the infringement procedure per department per year (cross-sectional time-series data). There is one observation per department per year per decision stage (2002-2020). The dataset uses current department names. 

\subheading{Variables}

\begin{description}[labelwidth=130pt, leftmargin=\dimexpr\labelwidth+\labelsep\relax, font=\normalfont, itemsep=10pt]
\item[\code{key\_id}] \code{numeric}\hspace{5pt}An ID number that uniquely identifies each observation in the dataset. 
\item[\code{year}] \code{numeric}\hspace{5pt}The year the decision was issued by the Commission.
\item[\code{department\_id}] \code{numeric}\hspace{5pt}An ID number that uniquely identifies each Commission department.
\item[\code{department}] \code{string}\hspace{5pt}The name of the Commission department that opened the infringement case.
\item[\code{department\_code}] \code{string}\hspace{5pt}A multi-letter code assigned by the Commission that uniquely identifies each department.
\item[\code{decision\_stage\_id}] \code{numeric}\hspace{5pt}An ID number that uniquely identifies each decision stage in the infringement procedure. Coded \code{1} for letters of formal notice under Article 258 of the Treaty on the Functioning of the European Union (TFEU), coded \code{2} for reasoned opinions under Article 258, coded \code{3} for referrals to the Court under Article 258, coded \code{4} for letters of formal notice under Article 260, coded \code{5} for reasoned opinions under Article 260, and coded \code{6} for referrals to the Court under Article 265
\item[\code{decision\_stage}] \code{string}\hspace{5pt}The decision stage of the infringement procedure. Possible values include: \code{Letter of formal notice (Article 258)}, \code{Reasoned opinion (Article 259)}, \code{Referral to the Court (Article 258)}, \code{Letter of formal notice (Article 260)}, \code{Reasoned opinion (Article 260)}, and \code{Referral to the Court (Article 260)}. 
\item[\code{count\_decisions}] \code{numeric}\hspace{5pt}A count of the number of decisions made by the Commission in infringement cases at this level of aggregation.
\end{description}
%--------------------------------------------------%
% dataset
%--------------------------------------------------%

\headerpage{decisions\_csts\_dp\_ct}{Decision-level cross-sectional time-series data by department and case type}{30}{12}

\subheading{Description}

This dataset includes aggregated data on the number of decisions per stage of the infringement procedure per department per year (cross-sectional time-series data) broken down by case type (noncommunication vs nonconformity). There is one observation per department per year per decision stage per case type (2002-2020). The dataset uses current department names. 

\subheading{Variables}

\begin{description}[labelwidth=130pt, leftmargin=\dimexpr\labelwidth+\labelsep\relax, font=\normalfont, itemsep=10pt]
\item[\code{key\_id}] \code{numeric}\hspace{5pt}An ID number that uniquely identifies each observation in the dataset. 
\item[\code{year}] \code{numeric}\hspace{5pt}The year the decision was issued by the Commission.
\item[\code{department\_id}] \code{numeric}\hspace{5pt}An ID number that uniquely identifies each Commission department.
\item[\code{department}] \code{string}\hspace{5pt}The name of the Commission department that opened the infringement case.
\item[\code{department\_code}] \code{string}\hspace{5pt}A multi-letter code assigned by the Commission that uniquely identifies each department.
\item[\code{case\_type\_id}] \code{numeric}\hspace{5pt}An ID number that uniquely identifies each type of state aid cases. Coded \code{1} for noncommunication cases, which are cases that relate to a member state failing to notify the Commission that it has transposed a directive by the stated deadline. Coded \code{2} for nonconformity cases, which are cases that relate to a member state incorrectly transposing a directive. 
\item[\code{case\_type}] \code{string}\hspace{5pt}The type of the infringement case. There are two types of cases. Coded \code{noncommunication} for cases that relate to a member state failing to notify the Commission that it has transposed a directive by the stated deadline. Coded \code{nonconformity} for cases that relate to a member state incorrectly transposing a directive. 
\item[\code{decision\_stage\_id}] \code{numeric}\hspace{5pt}An ID number that uniquely identifies each decision stage in the infringement procedure. Coded \code{1} for letters of formal notice under Article 258 of the Treaty on the Functioning of the European Union (TFEU), coded \code{2} for reasoned opinions under Article 258, coded \code{3} for referrals to the Court under Article 258, coded \code{4} for letters of formal notice under Article 260, coded \code{5} for reasoned opinions under Article 260, and coded \code{6} for referrals to the Court under Article 266
\item[\code{decision\_stage}] \code{string}\hspace{5pt}The decision stage of the infringement procedure. Possible values include: \code{Letter of formal notice (Article 258)}, \code{Reasoned opinion (Article 259)}, \code{Referral to the Court (Article 258)}, \code{Letter of formal notice (Article 260)}, \code{Reasoned opinion (Article 260)}, and \code{Referral to the Court (Article 260)}. 
\item[\code{count\_decisions}] \code{numeric}\hspace{5pt}A count of the number of decisions made by the Commission in infringement cases at this level of aggregation.
\end{description}
%--------------------------------------------------%
% dataset
%--------------------------------------------------%

\headerpage{decisions\_ddy}{Decision-level directed dyad-year data}{30}{12}

\subheading{Description}

This dataset includes aggregated data on the number of decisions per stage of the infringement procedure per department per member state per year (directed dyad-year data). There is one observation per department per member state per year per decision stage (2002-2020), excluding directed dyad-years where the state was not a member of the EU. The dataset uses current department names. 

\subheading{Variables}

\begin{description}[labelwidth=130pt, leftmargin=\dimexpr\labelwidth+\labelsep\relax, font=\normalfont, itemsep=10pt]
\item[\code{key\_id}] \code{numeric}\hspace{5pt}An ID number that uniquely identifies each observation in the dataset. 
\item[\code{year}] \code{numeric}\hspace{5pt}The year the decision was issued by the Commission.
\item[\code{department\_id}] \code{numeric}\hspace{5pt}An ID number that uniquely identifies each Commission department.
\item[\code{department}] \code{string}\hspace{5pt}The name of the Commission department that opened the infringement case.
\item[\code{department\_code}] \code{string}\hspace{5pt}A multi-letter code assigned by the Commission that uniquely identifies each department.
\item[\code{member\_state\_id}] \code{numeric}\hspace{5pt}An ID number that uniquely identifies each member state. This ID number is assigned when member states are sorted by accession date and then alphabetically. 
\item[\code{member\_state}] \code{string}\hspace{5pt}The name of the member state that the Commission opened the case against. 
\item[\code{member\_state\_code}] \code{string}\hspace{5pt}A two letter code assigned by the Commission that uniquely identifies each member state. 
\item[\code{decision\_stage\_id}] \code{numeric}\hspace{5pt}An ID number that uniquely identifies each decision stage in the infringement procedure. Coded \code{1} for letters of formal notice under Article 258 of the Treaty on the Functioning of the European Union (TFEU), coded \code{2} for reasoned opinions under Article 258, coded \code{3} for referrals to the Court under Article 258, coded \code{4} for letters of formal notice under Article 260, coded \code{5} for reasoned opinions under Article 260, and coded \code{6} for referrals to the Court under Article 267
\item[\code{decision\_stage}] \code{string}\hspace{5pt}The decision stage of the infringement procedure. Possible values include: \code{Letter of formal notice (Article 258)}, \code{Reasoned opinion (Article 259)}, \code{Referral to the Court (Article 258)}, \code{Letter of formal notice (Article 260)}, \code{Reasoned opinion (Article 260)}, and \code{Referral to the Court (Article 260)}. 
\item[\code{count\_decisions}] \code{numeric}\hspace{5pt}A count of the number of decisions made by the Commission in infringement cases at this level of aggregation.
\end{description}
%--------------------------------------------------%
% dataset
%--------------------------------------------------%

\headerpage{decisions\_ddy\_ct}{Decision-level directed dyad-year data by case type}{30}{12}

\subheading{Description}

This dataset includes aggregated data on the number of decisions per stage of the infringement procedure per department per member state per year (directed dyad-year data) broken down by case type (noncommunication vs nonconformity). There is one observation per department per member state per year per case type per decision stage (2002-2020), excluding directed dyad-years where the state was not a member of the EU. The dataset uses current department names. 

\subheading{Variables}

\begin{description}[labelwidth=130pt, leftmargin=\dimexpr\labelwidth+\labelsep\relax, font=\normalfont, itemsep=10pt]
\item[\code{key\_id}] \code{numeric}\hspace{5pt}An ID number that uniquely identifies each observation in the dataset. 
\item[\code{year}] \code{numeric}\hspace{5pt}The year the decision was issued by the Commission.
\item[\code{department\_id}] \code{numeric}\hspace{5pt}An ID number that uniquely identifies each Commission department.
\item[\code{department}] \code{string}\hspace{5pt}The name of the Commission department that opened the infringement case.
\item[\code{department\_code}] \code{string}\hspace{5pt}A multi-letter code assigned by the Commission that uniquely identifies each department.
\item[\code{member\_state\_id}] \code{numeric}\hspace{5pt}An ID number that uniquely identifies each member state. This ID number is assigned when member states are sorted by accession date and then alphabetically. 
\item[\code{member\_state}] \code{string}\hspace{5pt}The name of the member state that the Commission opened the case against. 
\item[\code{member\_state\_code}] \code{string}\hspace{5pt}A two letter code assigned by the Commission that uniquely identifies each member state. 
\item[\code{case\_type\_id}] \code{numeric}\hspace{5pt}An ID number that uniquely identifies each type of state aid cases. Coded \code{1} for noncommunication cases, which are cases that relate to a member state failing to notify the Commission that it has transposed a directive by the stated deadline. Coded \code{2} for nonconformity cases, which are cases that relate to a member state incorrectly transposing a directive. 
\item[\code{case\_type}] \code{string}\hspace{5pt}The type of the infringement case. There are two types of cases. Coded \code{noncommunication} for cases that relate to a member state failing to notify the Commission that it has transposed a directive by the stated deadline. Coded \code{nonconformity} for cases that relate to a member state incorrectly transposing a directive. 
\item[\code{decision\_stage\_id}] \code{numeric}\hspace{5pt}An ID number that uniquely identifies each decision stage in the infringement procedure. Coded \code{1} for letters of formal notice under Article 258 of the Treaty on the Functioning of the European Union (TFEU), coded \code{2} for reasoned opinions under Article 258, coded \code{3} for referrals to the Court under Article 258, coded \code{4} for letters of formal notice under Article 260, coded \code{5} for reasoned opinions under Article 260, and coded \code{6} for referrals to the Court under Article 268
\item[\code{decision\_stage}] \code{string}\hspace{5pt}The decision stage of the infringement procedure. Possible values include: \code{Letter of formal notice (Article 258)}, \code{Reasoned opinion (Article 259)}, \code{Referral to the Court (Article 258)}, \code{Letter of formal notice (Article 260)}, \code{Reasoned opinion (Article 260)}, and \code{Referral to the Court (Article 260)}. 
\item[\code{count\_decisions}] \code{numeric}\hspace{5pt}A count of the number of decisions made by the Commission in infringement cases at this level of aggregation.
\end{description}
%--------------------------------------------------%
% dataset
%--------------------------------------------------%

\headerpage{decisions\_net}{Decision-level network data}{30}{12}

\subheading{Description}

This dataset includes multi-dimensional network data on infringement decisions. There is one dimension per decision stage. Network data is similar to directed dyad-year data except that it only includes directed dyad-years with at least one decision with respect to each stage. For every year, there is one node per department and one node per member state. Edges can only exist between a department and a member state. There is an edge between a department and a member state with respect to each stage if and only if the department issued at least one decision against the member state during that year. The weight of the edge is the number of decisions that the department opened against the member state. There is one observation per department per member state per year per decision stage (2002-2020), excluding directed dyad-years where the state was not a member of the EU, but only if count of decisions is positive.

\subheading{Variables}

\begin{description}[labelwidth=130pt, leftmargin=\dimexpr\labelwidth+\labelsep\relax, font=\normalfont, itemsep=10pt]
\item[\code{key\_id}] \code{numeric}\hspace{5pt}An ID number that uniquely identifies each observation in the dataset. 
\item[\code{year}] \code{numeric}\hspace{5pt}The year the decision was issued by the Commission.
\item[\code{layer\_id}] \code{numeric}\hspace{5pt}An ID number that uniquely identifies each layer of the network.
\item[\code{layer}] \code{string}\hspace{5pt}The layer of the network, which is the decision stage.
\item[\code{from\_node\_id}] \code{numeric}\hspace{5pt}An ID number that uniquely identifies each node in the network that creates a link, which is always a Commission department.
\item[\code{from\_node}] \code{string}\hspace{5pt}The name of the Commission department that opened the infringement case.
\item[\code{to\_node\_id}] \code{numeric}\hspace{5pt}An ID number that uniquely identifies each node in the network that receives a link, which is always a member state.
\item[\code{to\_node}] \code{string}\hspace{5pt}The name of the member state that the Commission issued the decision against. 
\item[\code{edge\_weight}] \code{numeric}\hspace{5pt}The weight of the edge, which is the number of decisions opened by the Commission.
\end{description}
%--------------------------------------------------%
% dataset
%--------------------------------------------------%

\headerpage{decisions\_net\_ct}{Decision-level network data by case type}{30}{12}

\subheading{Description}

This dataset includes multi-dimensional network data on infringement decisions. There is one dimension per case type (noncommunication vs nonconformity) per decision stage. Network data is similar to directed dyad-year data except that it only includes directed dyad-years with at least one decision with respect to each case type and stage. For every year, there is one node per department and one node per member state. Edges can only exist between a department and a member state. There is an edge between a department and a member state with respect to each case type and stage if and only if the department issued at least one decision against the member state during that year. The weight of the edge is the number of decisions that the department opened against the member state. There is one observation per department per member state per year per case type per decision stage (2002-2020), excluding directed dyad-years where the state was not a member of the EU, but only if the count of decisions is positive.

\subheading{Variables}

\begin{description}[labelwidth=130pt, leftmargin=\dimexpr\labelwidth+\labelsep\relax, font=\normalfont, itemsep=10pt]
\item[\code{key\_id}] \code{numeric}\hspace{5pt}An ID number that uniquely identifies each observation in the dataset. 
\item[\code{year}] \code{numeric}\hspace{5pt}The year the decision was issued by the Commission.
\item[\code{d1\_layer\_id}] \code{numeric}\hspace{5pt}An ID number that uniquely identifies each layer of the network within the first dimension.
\item[\code{d1\_layer}] \code{string}\hspace{5pt}The layer of the network within the first dimension, which is type of case.
\item[\code{d2\_layer\_id}] \code{numeric}\hspace{5pt}An ID number that uniquely identifies each layer of the network within the second dimension.
\item[\code{d2\_layer}] \code{string}\hspace{5pt}The layer of the network within the second dimension, which is the decision stage.
\item[\code{from\_node\_id}] \code{numeric}\hspace{5pt}An ID number that uniquely identifies each node in the network that creates a link, which is always a Commission department.
\item[\code{from\_node}] \code{string}\hspace{5pt}The name of the Commission department that opened the infringement case.
\item[\code{to\_node\_id}] \code{numeric}\hspace{5pt}An ID number that uniquely identifies each node in the network that receives a link, which is always a member state.
\item[\code{to\_node}] \code{string}\hspace{5pt}The name of the member state that the Commission opened the case against. 
\item[\code{edge\_weight}] \code{numeric}\hspace{5pt}The weight of the edge, which is the number of decisions opened by the Commission.
\end{description}

%--------------------------------------------------%
% end document
%--------------------------------------------------%

\end{flushleft}

\end{document}
